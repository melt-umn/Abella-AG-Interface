\documentclass[11pt]{article}

\usepackage[letterpaper,margin=1in]{geometry}

\usepackage[hidelinks]{hyperref}

\usepackage{color}
\definecolor{orange}{rgb}{1.0,0.59,0}

\usepackage{listings}

\lstset{
  language=C,
  basicstyle=\ttfamily,
  morekeywords={
    Extensible_Theorem, Theorem, Existing_Theorem,
    Grammar,
    on
  },
  keywordstyle=\color{blue}\ttfamily,
  morekeywords={
    skip
  },
  stringstyle=\color{green}\ttfamily
}


%reference a given section
\renewcommand{\sec}[1]{\hyperref[sec:#1]{Section~\ref*{sec:#1}}}
%reference a given subsection
\newcommand{\subsec}[1]{\hyperref[subsec:#1]{Subsection~\ref*{subsec:#1}}}
%reference a given equation
\newcommand{\eqn}[1]{\hyperref[eqn:#1]{equation~\ref*{eqn:#1}}}
%reference a given figure
\newcommand{\fig}[1]{\hyperref[fig:#1]{Figure~\ref*{fig:#1}}}


\newcommand{\grammar}[1]{\texttt{#1}}
\newcommand{\attr}[1]{\texttt{#1}}


\newcommand{\tyarrow}{\ensuremath{\rightarrow}}


%to make Eric happy
\newcommand{\silver}{\textsc{Silver}}


\newcommand{\enote}[1]{{\color{blue} \textbf{EVW:} \emph{#1}}}
\newcommand{\enoteres}[1]{}
\newcommand{\dnote}[1]{{\color{orange} \textbf{DM:} \emph{#1}}}
\newcommand{\dnoteres}[1]{}
\newcommand{\todo}[1]{{\color{red} #1}}


\begin{document}





\dnote{This document was originally written assuming \silver{} would
  generate a specification file and a theorem.  This idea has changed
  to having \silver{} generate a specification file and a sort of
  interface file for theorems instead.  However, much of the document
  has not been rewritten to reflect this, so some parts may no longer
  be valid.}


\section{Introduction}

Recall briefly the requirements we have laid out previously for
proving properties of extensible languages encoded as attribute
grammars.  Language properties are proven by induction on a decorated
tree of a particular nonterminal type.  For host-introduced
properties, each language component must provide a proof of the
property in the inductive case for each production it introduces.  For
extension-introduced properties, the procedure depends on the
nonterminal on which induction is being done.  If the nonterminal was
introduced by the extension, the extension must provide a proof case
for all the productions it introduced for the nonterminal.  If the
nonterminal was introduced by the host, the extension must provide a
proof for all the productions it knows (both host- and
extension-introduced productions) as well as prove preservability of
the property.


As we did above, we often discuss host languages and extensions when
talking about \silver{}, as \silver{} is designed to write extensible
languages with this structure.  However, languages written in
\silver{} do not actually consist of a host language with extensions.
All \silver{} has are grammars.
%
For example, we might initially think a host language doesn't import
any grammars, but it probably imports \grammar{silver:core}.
%
We might also have a language which is truly composed of a host
language and extensions to it, but then decide to build another
extension on top of an existing extension or extensions.
%
For these reasons we cannot design the composition system of the
prover to work on ``host'' grammars and ``extension'' grammars, only
on \emph{grammars}, which may have host-like and extension-like
features to them.
%
Despite these complications, we can still use our theory to construct
a theorem prover to prove properties of \silver{} languages.  Because
extension-introduced properties on extension-introduced nonterminals
are treated the same as host-introduced properties on host-introduced
nonterminals, we can essentially treat all grammars as extensions for
constructing proofs.


Our prover works by having \silver{} compile a grammar to an Abella
specification.  The grammar's author then uses a theorem prover
interface designed to permit the user to prove properties about this
Abella specification in a way that looks and feels somewhat like
\silver, with decorated trees, attribute accesses, and other \silver{}
constructs.  This interface also handles the proof requirements for
the theorems which arise from extensibility, such as ensuring proofs
are given for new productions for theorems from imported grammars.
%
Our purpose here is to address the handling of extensibility of
\silver{} languages in the processes of compilation by \silver{} and
proving properties using the interface.




\section{Generating Specifications} \label{sec:generation}

In generating specifications, we want only to generate each
non-component piece of the specification (e.g. full relations, encoded
productions) once.  Therefore we generate the specification for each
grammar to output the (full) encoded versions of the constructs it
declares only, and none of those imported, as follows:
\begin{itemize}
\item Declared nonterminal:  Output encoded nonterminal type, node
  type, node tree constructor, full structure equality relation, full
  forwarding equation, full WPD relation, and full WPD node relation.
\item Declared production:  Output encoded production constructor.
\item Declared attribute:  Output \texttt{is\_inherited} property if
  the attribute is inherited.
\item Declared attribute occurrence:  Output access relation and full
  equation relation
\end{itemize}
The component relations are output for the constructs which are
introduced by the current grammar, since they are grammar-specific and
can't be replicated accidentally.


Another issue we find in generating specifications is that we cannot
actually import one Abella specification we generate into another
because we declare constants of a result type \texttt{prop}.  For
example, our access relations have type
\[\texttt{\textit{nonterminal} \tyarrow{} \textit{node} \tyarrow{} \$attrVal
  (\textit{attr type}) \tyarrow{} prop}\]
%
For some reason, Abella disallows importing files containing such
constants despite allowing them to be used in single files.  The
interface gets around this by simply ``replaying'' the file into the
background Abella process (essentially copying its text into the
Abella process).  This isn't possible in the specifications, which are
supposed to be written in straight Abella, which must use regular
Abella imports.  Therefore the specification files output by
compilation are not valid Abella (cannot be processed by Abella alone)
if the \silver{} specification imports another grammar, since we will
not output anything in the specification for the import, relying
instead on it being imported in the theorem file.
\dnote{I might open an Abella bug and see if I can get this to change,
  but I don't want to open a bug and have them decide they don't need
  \texttt{prop}-typed constants at all.}





\section{Generating Theorem Files}

One of the primary issues in handling extensibility is to ensure that
each grammar provides not only proofs of its own properties, but also
that it provides proofs in accordance with the grammars it imports.
This requires either providing a way to check after the fact that the
proofs have been provided satisfactorily, or a way to generate the
correct set-up beforehand.  I prefer the latter idea, as the user
essentially gets a skeleton of the necessary proofs to provide.  This
also would let us put the correct imports in the file.  This is
beneficial because they need to happen in the correct order, as each
specification file is only acceptable if the specifications it imports
are imported prior to its own import (discussed above in
\sec{generation}).


An alternative which we will mention here would be to have the
interface generate the necessary theorems on the fly.  It would import
the theorem files for the imported grammar in addition to the
specification files and generate automatic goals for the new proof
obligations.  The theorems would never be entered in the file, only
the proofs.
%
This seems like a poor choice for several reasons.  One reason is that
the theorems being proven would never be shown in the file, only the
specific proof cases being done.  It would also make it confusing when
the user is allowed to add his new theorems for the new grammar, as
they would only be allowed between certain automatically-generated
goals.  Finally, this approach, as opposed to the theorem-file
generation approach, would require the work of gathering the imported
theorems and ordering them every time, rather than just once at
compilation time.
%
For these reasons, we discard this method and continue our discussion
with the theorem-file generation method in mind, although many of the
same issues apply.


When compiling a grammar, then, we should not only output the
specification, but also a file with the necessary theorems from the
imported grammars.  This requires both finding the theorem files for
the imported grammars and combining the theorems in an acceptable
order from these imports.  We address these in turn before looking at
how the interface needs to handle extensibility while proving.




\subsection{Finding Imported Theorem Files} \label{subsec:findingThmFiles}

In order to generate a theorem file, we need to find the theorem files
for the grammars which are imported.  This is necessary because we
need to know the theorems they introduce to ensure the current grammar
fulfills its proof obligations in regards to the properties introduced
by the grammars it imports.  We want to do this by placing these
theorems in the theorem file for the current grammar.


I'm not sure how to handle finding the theorem files.  We could
require the developments be put in with the grammar and have a
particular name for the file based on the grammar name
(e.g. \texttt{a:b:c\_thms.svthm} for grammar \grammar{a:b:c}).  This
would let us find it if it exists, since we would know \emph{where} to
look and for \emph{what} we are looking.  Perhaps there are better
ideas which rely less on conventions.  The main requirement for a good
solution to this problem are, as I see it, is an ability to handle a
large number of imported grammars graciously.  For example, if our
current grammar imports five others, we don't want to have to type out
the locations of five theorem files.


In a similar vein, we may want to have a particular location for the
specification file for a grammar.  This would make generating a
theorem file easier, since we would not need to search for a
specification file.  A gracious solution for identifying theorem files
for imports would probably also be a gracious solution for finding the
corresponding specification file.




\subsection{Combining Theorems from Multiple Imported Grammars} \label{subsec:ordering}

The order theorems are proven in a grammar is important because later
proofs may rely on earlier theorems being used as lemmas.  When we
take the theorems from multiple grammars, we need to combine them
while maintaining the ordering from each grammar.  This is necessary
for soundness of the composition of the proofs done in the imported
grammars and the proofs done in the importing grammar, especially when
composed with proofs from independent grammars which start with the
same (quasi-)host grammar as a base.  By making consistent orderings
in all grammars we ensure the composition will succeed.


To combine the theorems from imported grammars into a coherent order,
we need an idea of the import structure of the grammars.  If we import
\grammar{A}, \grammar{B}, and \grammar{C}, but \grammar{A} also
imports \grammar{C}, then all of \grammar{C}'s theorems should already
be included in \grammar{A}'s theorems in its theorem file, and we only
need to combine the theorems from \grammar{A} and \grammar{B}.  If
they have a common base of theorems from an imported grammar,
everything between the same theorems from the common base needs to be
placed before the next shared theorem.  That is, if \grammar{A} has
theorems $S_1$, $A_1$, $S_2$ and \grammar{B} has theorems $S_1$,
$B_1$, $B_2$, $S_2$, then $A_1$, $B_1$, and $B_2$ all need to be
placed in the ordering after $S_1$ and before $S_2$.  The same goes
for everything before the first shared theorem and everything after
the last shared theorem.  If there is no shared base, the theorems can
be regarded as all coming after the last shared one.


The theorems from grammars imported together but which have no
relationship with one another have no order between them.  In our
example from above, $A_1$ is neither inherently earlier nor later than
$B_1$.  This suggests we could put them in any order, so long as we
respected the order from each grammar.
%
If they were pure extensions and this were the final composition, that
would be the case.  However, since this is simply another grammar, we
need to ensure that the new pieces of the proof for the theorems
brought in from other grammars do not rely on each other in a way
which will not allow a sound composition with other grammars which
also import both.
%
Suppose grammars \grammar{G} and \grammar{H} both import \grammar{A}
and \grammar{B}.  Suppose \grammar{G} puts $A_1$ before $B_1$ and
proves its obligations for $B_1$ by relying on $A_1$ and \grammar{H}
puts $B_1$ before $A_1$ and proves its obligations for $A_1$ by
relying on $B_1$.  Since these pieces of the proofs relied on the
opposite order, the composition cannot succeed.


The obvious question is whether MWDA makes such a dependence between
theorems originating in independent grammars impossible.  It seems at
first like it ought to make it impossible, since it prevents reliance
of existing synthesized attributes on new inherited attributes.  This
does not, however, mean no dependence can be introduced by the
theorems.  Suppose $A_1$ is ``about'' the values an attribute \attr{a}
introduced by \grammar{A} can take and $B_1$ is similarly ``about''
the values an attribute \attr{b} introduced by \grammar{B} can take.
If \attr{b} is defined on a production in the importing grammar in
terms of \attr{a}, a valid equation under MWDA if the flow type of
\attr{a} is a subset of the flow type of \attr{b} (which means
\attr{a} cannot rely on any inherited attributes introduced by
\grammar{A}, only inherited attributes from a shared underlying
grammar or nothing), then the proof of $B_1$ can rely on $A_1$.
%
This dependence is only possible because we are building grammars on
top of other, separate grammars, which is part of \silver{}'s only
having grammars, not host languages and extensions.

To avoid having inconsistent ordering dependencies in compositions, as
unlikely as it seems someone will introduce such a dependency, we need
to place the theorems consistently between different grammars which
import the same grammars.  This ensures that any reliance on ordering
from different grammars still has the same ordering in the
composition.  We will ensure the ordering is consistent by placing the
theorems in order alphabetically by grammar name.
%
If a grammar imports grammars \grammar{A} and \grammar{B} as described
above, we will place their theorems in the order $A_1$, $B_1$, $B_2$.
We can be certain any other grammar importing both will also place the
theorems in the same order, so the composition can place them in the
same order for consistency.  If the composition also includes another
grammar \grammar{Another} with theorems $Another_1$ and $Another_2$,
the composition will have theorems in the order $A_1$, $Another_1$,
$Another_2$, $B_1$, $B_2$.  This does not interfere with any
dependencies because the earlier theorems are still proven before
being used, just with some intervening proofs done as well.

Note that placing the theorems in the file to do the new proof pieces
required in the new grammar means that we do \emph{not} need to import
the theorem files from imported grammars at the time of proving the
properties with the interface.  The only content they hold that we do
place in the generated theorem file for the importing grammar is the
actual proofs, which are not relevant to the new grammar.





\section{Avoiding Overwriting Work}

In generating the theorem file, we need to be careful to avoid
overwriting someone's work.  If \grammar{B} imports \grammar{A} and I
am proving theorems about \grammar{B} and realize I need to add a
theorem to \grammar{A}, I don't want to regenerate the theorem file
for \grammar{B} and accidentally lose all the work I have done.  This
is a concern peculiar to the theorem file, as the specification file
should never be edited by the user.


To avoid accidental overwriting, we might do one of the following:
\begin{itemize}
\item Refuse to generate a theorem file if one is found.  This
  requires the user to remove his work himself, probably by copying it
  elsewhere, then integrating it into the new file.
\item Ask the user for permission to overwrite.  This informs the user
  his work will be lost if he says yes, and makes any loss his own
  fault.
\item Write to a new file with a different name.  This makes the
  skeleton file available without destroying any work.
\end{itemize}
All these sort of assume the theorem file has a set location, which is
suggested in \subsec{findingThmFiles} but not settled on.  Perhaps we
will decide it doesn't, in which case the solutions to this issue
might change as well.





\section{Proving Process}

\dnote{These last two sections have been rewritten to reflect the idea
  of using generated interface files to inform the prover interface
  rather than directly generating the theorem file.}

With the imported grammars' theorems placed in order in the interface
file, as well as the correct specification imports, the user can start
proving. He does this not only by fulfilling his proof obligations for
the theorems which sit in the interface file, but also by adding
theorems for the current grammar.  These new theorems can go between
any existing theorems, at the begining, or at the end. They can also
be made mutual with existing theorems.


The interface needs to keep track of what needs to be proven for each
theorem.
%
If we are in grammar \grammar{E} which imported grammar \grammar{H}
and we have theorem $T$ from \grammar{H}, we need to provide proofs
for $T$ only for the productions which \grammar{E} introduces for the
theorem's induction nonterminal.
%
If we have theorem $U$ introduced by \grammar{E} on a nonterminal
introduced by \grammar{E}, we only need cases for the productions
building this nonterminal.
%
If theorem $V$ introduced by \grammar{E} is on a nonterminal
introduced by \grammar{H} which \grammar{E} imported, \grammar{E}
needs to do proofs for the cases for all productions which it knows as
well as prove preservability.


An obvious question is whether we could generate these proof
obligations for imported theorems as part of the interface file
generation rather than have the interface determine them on the fly.
The answer is yes, we could generate them then, with an extra note in
the interface file for the new cases to be proven.
%
However, determining the proof cases required for imported theorems is
similar to determining the proof cases required for newly-introduced
theorems.  I don't think it would save any work in writing the prover
interface and would require duplicating some of that work in the
Silver extension for generating the interface file.
%
For these reasons, we are inclined to leave the generation of the
particular cases entirely to the prover interface, not adding anything
extra to indicate them in the interface file.


The prover interface can determine the correct proof obligations by
keeping track of the current grammar in which the proofs are being
done, the grammar introducing each theorem, and the grammar
originating different productions and nonterminals.
%
We can track the current grammar by adding some sort of a grammar
declaration to the theorem file.  This can be the same construct which
informs the prover interface which theorem interface file to use in
checking the correctness of the proofs.  This could be something like
\begin{lstlisting}
  Grammar "gra:mm:ar".
\end{lstlisting}
%
The grammar originating each theorem could be part of the interface
file.  Any theorems not coming from the interface file would be
assumed to be from the current grammar.
%
The grammars originating productions and nonterminals could be found
by which grammar's specification we were reading when we found the
type or constructor declared.





\section{Imported Theorems Not Requiring New Proof Cases}

In the preceding discussion we have somewhat implicitly been
discussing extensible theorems, those proven by induction on a tree of
a particular type, and assuming the new grammar was adding new
productions to the induction type, resulting in new proof obligations
for all the imported theorems.  New productions are not always added,
nor are extensible theorems our only type of theorems.

Regardless of the type of the theorem, we can still order it as
described in \subsec{ordering}.
%
For extensible theorems with no new proof obligations, the interface
file generation can still enter the theorem as it otherwise would,
with the prover interface determining at the time it processes the
theorem declaration that there is nothing to prove and immediately
finishing it.


We also have non-extensible theorems.  One common use of these is to
prove properties of the behavior of a function.  We might also have
these arise as corollaries to extensible theorems, where we proved the
general case of a theorem and are interested in a specific case or
where we might combine two results proven as extensible theorems to
prove another result as a direct consequence.
%
Because the full proofs of these theorems are given when they are
introduced, there is nothing to be done in a grammar which imports the
theorem.  These still need to be introduced by the prover interface
into Abella to allow future use, but the details of how this is to be
done depends on the details of how the interface system is going to
work, which are not set yet.





\end{document}


